\documentclass[paper=a4, fontsize=11pt]{scrartcl} % A4 paper and 11pt font size


\usepackage{graphicx} %Paquete de graficos
\usepackage[english]{babel} % English language/hyphenation
\usepackage{amsmath,amsfonts,amsthm} % Math packages


allsectionsfont{\centering \normalfont\scshape} % Make all sections centered, the default font and small caps

%----------------------------------------------------------------------------------------
%	TITLE SECTION
%----------------------------------------------------------------------------------------

\newcommand{\horrule}[1]{\rule{\linewidth}{#1}} % Create horizontal rule command with 1 argument of height

\title{
\normalfont \normalsize 
\textsc{Universidad de los Andes} \\ [25pt]
\horrule{0.5pt} \\[0.4cm]
\huge Metodos Computacionales - Tarea 3  \\ 
\horrule{2pt} \\[0.5cm] 
}

\author{Juan Sebastian Duran Torres} 

\date{20 de Julio de 2017} 

\begin{document}

\maketitle


\section{Onda en dos dimensiones}

A partir de la solución de la ecuación de onda en dos dimensiones y las condiciones dadas se construyeron las siguientes graficas en dos y tres dimensiones respectivamente, la simulación dura aproximadamente 60 segundos.
\subsection{ 2D} 
A continuacion se  muestran las graficas para el estado de la onda desde una vista de planta.
\begin{figure}[h]
	\centering
	\includegraphics[width=\textwidth]{Onda2D_t30.png}
	\caption{Estado de la onda para un tiempo de 30 segundos}
\end{figure}

\begin{figure}[h]
	\centering
	\includegraphics[width=\textwidth]{Onda2D_t60.png}
	\caption{Estado de la onda para un tiempo de 60 segundos}
\end{figure}


\subsection{3D}
A continuacion se muestran las graficas para el estado de la onda en 3 dimensiones

\begin{figure}[h]
	\centering
	\includegraphics[width=\textwidth]{Onda3D_t30.png}
	\caption{Estado de la onda para un tiempo de 30 segundos}
\end{figure}

\begin{figure}[h]
	\centering
	\includegraphics[width=\textwidth]{Onda3D_t60.png}
	\caption{Estado de la onda para un tiempo de 60 segundos}
\end{figure}



\section{Sistema solar}

A partir de la solución de la ecuación de movimiento para los planetas e implementando el metodo de Leap-frog se calcularon las orbitas de los planetas del sistema solar y el sol para un periodo de tiempo de 255 años.
\begin{figure}[h]
	\centering
	\includegraphics[width=\textwidth]{orbitas.png}
	\caption{Orbitas de los planetas del sistema solar}
\end{figure}


\end{document}


